\documentclass[journal]{IEEEtran}
\usepackage{blindtext}
\usepackage{graphicx}


% *** CITATION PACKAGES ***
%
\usepackage{cite}
% cite.sty was written by Donald Arseneau
% V1.6 and later of IEEEtran pre-defines the format of the cite.sty package
% \cite{} output to follow that of IEEE. Loading the cite package will
% result in citation numbers being automatically sorted and properly
% "compressed/ranged". e.g., [1], [9], [2], [7], [5], [6] without using
% cite.sty will become [1], [2], [5]--[7], [9] using cite.sty. cite.sty's
% \cite will automatically add leading space, if needed. Use cite.sty's
% noadjust option (cite.sty V3.8 and later) if you want to turn this off.
% cite.sty is already installed on most LaTeX systems. Be sure and use
% version 4.0 (2003-05-27) and later if using hyperref.sty. cite.sty does
% not currently provide for hyperlinked citations.
% The latest version can be obtained at:
% http://www.ctan.org/tex-archive/macros/latex/contrib/cite/
% The documentation is contained in the cite.sty file itself.






% *** GRAPHICS RELATED PACKAGES ***
%
\ifCLASSINFOpdf{}
  % \usepackage[pdftex]{graphicx}
  % declare the path(s) where your graphic files are
  % \graphicspath{{../pdf/}{../jpeg/}}
  % and their extensions so you won't have to specify these with
  % every instance of \includegraphics
  % \DeclareGraphicsExtensions{.pdf,.jpeg,.png}
\else
  % or other class option (dvipsone, dvipdf, if not using dvips). graphicx
  % will default to the driver specified in the system graphics.cfg if no
  % driver is specified.
  % \usepackage[dvips]{graphicx}
  % declare the path(s) where your graphic files are
  % \graphicspath{{../eps/}}
  % and their extensions so you won't have to specify these with
  % every instance of \includegraphics
  % \DeclareGraphicsExtensions{.eps}
\fi
% graphicx was written by David Carlisle and Sebastian Rahtz. It is
% required if you want graphics, photos, etc. graphicx.sty is already
% installed on most LaTeX systems. The latest version and documentation can
% be obtained at:
% http://www.ctan.org/tex-archive/macros/latex/required/graphics/
% Another good source of documentation is "Using Imported Graphics in
% LaTeX2e" by Keith Reckdahl which can be found as epslatex.ps or
% epslatex.pdf at: http://www.ctan.org/tex-archive/info/
%
% latex, and pdflatex in dvi mode, support graphics in encapsulated
% postscript (.eps) format. pdflatex in pdf mode supports graphics
% in .pdf, .jpeg, .png and .mps (metapost) formats. Users should ensure
% that all non-photo figures use a vector format (.eps, .pdf, .mps) and
% not a bitmapped formats (.jpeg, .png). IEEE frowns on bitmapped formats
% which can result in "jaggedy"/blurry rendering of lines and letters as
% well as large increases in file sizes.
%
% You can find documentation about the pdfTeX application at:
% http://www.tug.org/applications/pdftex


% *** SPECIALIZED LIST PACKAGES ***
%
%\usepackage{algorithmic}
% algorithmic.sty was written by Peter Williams and Rogerio Brito.
% This package provides an algorithmic environment for describing algorithms.
% You can use the algorithmic environment in-text or within a figure
% environment to provide for a floating algorithm. Do NOT use the algorithm
% floating environment provided by algorithm.sty (by the same authors) or
% algorithm2e.sty (by Christophe Fiorio) as IEEE does not use dedicated
% algorithm float types and packages that provide these will not provide
% correct IEEE style captions. The latest version and documentation of
% algorithmic.sty can be obtained at:
% http://www.ctan.org/tex-archive/macros/latex/contrib/algorithms/
% There is also a support site at:
% http://algorithms.berlios.de/index.html
% Also of interest may be the (relatively newer and more customizable)
% algorithmicx.sty package by Szasz Janos:
% http://www.ctan.org/tex-archive/macros/latex/contrib/algorithmicx/




% *** ALIGNMENT PACKAGES ***
%
\usepackage{array}
% Frank Mittelbach's and David Carlisle's array.sty patches and improves
% the standard LaTeX2e array and tabular environments to provide better
% appearance and additional user controls. As the default LaTeX2e table
% generation code is lacking to the point of almost being broken with
% respect to the quality of the end results, all users are strongly
% advised to use an enhanced (at the very least that provided by array.sty)
% set of table tools. array.sty is already installed on most systems. The
% latest version and documentation can be obtained at:
% http://www.ctan.org/tex-archive/macros/latex/required/tools/



% *** SUBFIGURE PACKAGES ***
\usepackage[tight,footnotesize]{subfigure}
% subfigure.sty was written by Steven Douglas Cochran. This package makes it
% easy to put subfigures in your figures. e.g., "Figure 1a and 1b". For IEEE
% work, it is a good idea to load it with the tight package option to reduce
% the amount of white space around the subfigures. subfigure.sty is already
% installed on most LaTeX systems. The latest version and documentation can
% be obtained at:
% http://www.ctan.org/tex-archive/obsolete/macros/latex/contrib/subfigure/
% subfigure.sty has been superceeded by subfig.sty.



% *** FLOAT PACKAGES ***
%
%\usepackage{fixltx2e}
% fixltx2e, the successor to the earlier fix2col.sty, was written by
% Frank Mittelbach and David Carlisle. This package corrects a few problems
% in the LaTeX2e kernel, the most notable of which is that in current
% LaTeX2e releases, the ordering of single and double column floats is not
% guaranteed to be preserved. Thus, an unpatched LaTeX2e can allow a
% single column figure to be placed prior to an earlier double column
% figure. The latest version and documentation can be found at:
% http://www.ctan.org/tex-archive/macros/latex/base/


% *** PDF, URL AND HYPERLINK PACKAGES ***
%
\usepackage{url}
% url.sty was written by Donald Arseneau. It provides better support for
% handling and breaking URLs. url.sty is already installed on most LaTeX
% systems. The latest version can be obtained at:
% http://www.ctan.org/tex-archive/macros/latex/contrib/misc/
% Read the url.sty source comments for usage information. Basically,
% \url{my_url_here}.



% *** UTF-8 Characters ***
\usepackage[utf8]{inputenc}
\usepackage{amsmath}
\usepackage{amsthm}

% correct bad hyphenation here
\hyphenation{}


\begin{document}

\newtheorem{definition}{Definition}
\newtheorem{theorem}[definition]{Satz}
\newtheorem{corollar}[definition]{Korollar}

\renewcommand{\proofname}{Beweis}

\title{Satz von Rice: Grundlagen, Beweis und Implikationen}


\author{René~Filip$^{\dag}$\thanks{$^{\dag}$DHBW Karlsruhe, TINF13B1, Seminar Theoretische Informatik 2016}}


\maketitle


\begin{abstract}
Der Satz von Rice hat in der Informatik weitreichende Konsequenzen, denn er sagt aus, dass es kein allgemeines Computerprogramm gibt, das überprüft, ob der eigene Code eine gewünschte Funktion berechnet oder nicht. Nachdem wir die Grundlagen und Definitionen geklärt haben, beweisen wir den Satz mittels ``Many-to-One" Reduktion auf ein einfacheres Problem, für das sich ein direkter Beweis finden lässt.
\end{abstract}


\begin{IEEEkeywords}
  Satz von Rice, Theoretische Informatik, Entscheidbarkeit, Halteproblem.
\end{IEEEkeywords}



\section{Einführung}

  % Motivation
  % Was ist der Satz von Rice
  % Welche Folgen hat es?

\section{Grundlagen und Definitionen}

  \begin{definition}[Turingmaschine]
    Eine Turingmaschine (TM) ist definiert durch ein Septuple $(Q, \Gamma, B, \Sigma, \delta, q_0, F)$:
    \begin{enumerate}
      \item endliche Zustandsmenge $Q$
      \item endliche Bandalphabet $\Gamma$
      \item Leerzeichen $B \in \Gamma$
      \item endliche Eingabealphabet $\Sigma \subseteq \Gamma \setminus B$
      \item Zustandsübergangsfunktion \\$\delta \colon Q \times \Gamma \to Q \times \Gamma \times (L, R, N)$
      \item Anfangszustand $q_0 \in Q$
      \item Menge der akzeptierenden Endzustände $F \subseteq Q$
    \end{enumerate}
  \end{definition}

  Zwar kann eine Turingmaschine jedes Computerprogramm simulieren~\cite{Wegener2013:2.3}, doch kann man den Programmcode (sprich die Übergangsfunktion $\delta$) nicht ändern. Ein Computer nach der Von-Neumann-Architektur (VNA) lässt sich stattdessen programmieren. Um den Satz von Rice und dessen Beweis sauber zu formulieren, müssen wir zunächst eine ``universelle Turingmaschine" definieren:

  \begin{definition}[Universelle Turingmaschine]
    ...
  \end{definition}

  Die Universelle Turingmaschine nimmt nun als Eingabe nicht nur ein Wort $w \in \{0, 1\}^*$, sondern auch ein Computerprogramm bzw. Gödelnummer $\langle M \rangle$. Die Gödelnummer ist eine Repräsentationsart einer Turingmaschine und ist wie folgt definiert:

  \begin{definition}[Gödelnummer]
    Sei $M$ eine Turingmaschine $(\{q_1, \dotsc, q_t\}, \{0, 1, B\}, B, \{0, 1\}, \delta, q_1, \{q_2\})$. Außerdem sei $X_1 = 0$, $X_2 = 1$, $X_3 = B$, $D_1 = L$, $D_2 = N$ und $D_3 = R$. Nun kann die $z$-te Zeile der Übergangsfunktions-Tabelle $\delta$ der Länge $s$ wie folgt beschrieben werden:
    \begin{equation*}
      \delta(q_i, X_j) = (q_k, X_l, D_m) \Longrightarrow \operatorname{code}(z) := 0^i 10^j 10^k 10^l 10^m
    \end{equation*}
    wobei $i, k \ge 1$ und $j, l, m \in \{1, 2, 3\}$. Die Gödelnummer $\langle M \rangle$ der Turingmaschine $M$ ist nun definiert als
    \begin{equation*}
      \langle M \rangle := 111 \operatorname{code}(1) 11 \operatorname{code}(2) 11 \dotsc 11 \operatorname{code}(s) 111
    \end{equation*}
  \end{definition}

  Wir können leicht sehen, dass jedes Programm mit $111$ anfängt und endet. Jede ``Programmcode-Zeile" wird durch $11$ separiert. $\operatorname{code}(z)$ enthält niemals den Teilstring $11$. \cite{Wegener2013:2.4} beschreibt die Gödelnummer näher.

  \begin{definition}[Entscheidbarkeit]
    Eine Sprache $L$ heißt rekursiv (oder entscheidbar), wenn es eine Turingmaschine gibt, die auf allen Eingaben stoppt und die Eingabe $w$ genau dann akzeptiert, wenn $w \in L$ ist.~\cite{Wegener2013:2.1.2}
  \end{definition}

  Eine Turingmaschine, die für jede Eingabe anhält, nennt man auch Entscheider.

\section{Satz von Rice}

  %Satz
  %Was es sagt, Konsequenzen, Erläuterungen

  ...

  \begin{theorem}[Satz von Rice]\label{satz_von_rice}
    Sei $\mathcal{R}$ die Menge der von Turingmaschinen berechenbaren Funktionen und S eine Teilmenge von $S$  mit $S \ne \emptyset$ und $S \ne \mathcal{R}$, das heißt $S$ ist eine nicht triviale Teilmenge aller berechenbaren Funktionen. Dann ist die Sprache
    \begin{equation*}
      L(S) := \{\langle M \rangle \ \vert \ M \ \text{berechnet eine Funktion aus} \ S\}
    \end{equation*}
    nicht rekursiv.
  \end{theorem}

  Für den Beweis von \ref{satz_von_rice} siehe \ref{proof_rice}.

  ...

  \subsection{Beweisstrategie}

  \subsection{Aufbauende Sätze}

    \begin{theorem}
      $D$ ist nicht rekursiv.
    \end{theorem}
    \begin{proof}
      ...
    \end{proof}

    \begin{corollar}
      $\overline{D}$ ist nicht rekursiv.
    \end{corollar}
    \begin{proof}
      ...
    \end{proof}

    \begin{theorem}
      $H$ ist nicht rekursiv.
    \end{theorem}
    \begin{proof}
      ...
    \end{proof}

    \begin{theorem}
      $H_\varepsilon$ ist nicht rekursiv.
    \end{theorem}
    \begin{proof}
      ...
    \end{proof}

  \subsection{Beweis des Satzes von Rice}\label{proof_rice}

    \begin{proof}
      ...
    \end{proof}


% if have a single appendix:
%\appendix[Proof of the Zonklar Equations]
% or
%\appendix  % for no appendix heading
% do not use \section anymore after \appendix, only \section*
% is possibly needed

% use appendices with more than one appendix
% then use \section to start each appendix
% you must declare a \section before using any
% \subsection or using \label (\appendices by itself
% starts a section numbered zero.)
%


% \appendices{}
% \section{Proof of the First Zonklar Equation}
% use section* for acknowledgement
% \section*{Acknowledgment}
% The authors would like to thank\ldots


% Can use something like this to put references on a page
% by themselves when using endfloat and the captionsoff option.
\ifCLASSOPTIONcaptionsoff{}
  \newpage
\fi



% trigger a \newpage just before the given reference
% number - used to balance the columns on the last page
% adjust value as needed - may need to be readjusted if
% the document is modified later
%\IEEEtriggeratref{8}
% The "triggered" command can be changed if desired:
%\IEEEtriggercmd{\enlargethispage{-5in}}

% references section

% can use a bibliography generated by BibTeX as a .bbl file
% BibTeX documentation can be easily obtained at:
% http://www.ctan.org/tex-archive/biblio/bibtex/contrib/doc/
% The IEEEtran BibTeX style support page is at:
% http://www.michaelshell.org/tex/ieeetran/bibtex/
% \bibliographystyle{IEEEtran}
% argument is your BibTeX string definitions and bibliography database(s)
%

\bibliographystyle{IEEEtran}
\bibliography{./references}




% that's all folks
\end{document}
